\documentclass[a4paper,12pt]{article}

% Document
\usepackage[english,russian]{babel}
\usepackage[colorlinks,linkcolor=blue]{hyperref}

% Elements
\usepackage{tabularx}
\usepackage{xcolor}
\usepackage{tikz}

% Add theme
\makeatletter
\renewcommand\maketitle{{\raggedright
    \begin{tikzpicture}[remember picture,overlay]
        \node [
            shading = axis,
            rectangle,
            left color=color_primary,
            right color=color_primary!40!color_secondary,
            shading angle=30,
            anchor=north,
            minimum width=\paperwidth,
            minimum height=\paperheight
        ] (box) at (current page.north){};
    \end{tikzpicture}
    \vspace*{\fill}
        \begin{center}
            \textcolor{white}{\Huge \bfseries \sffamily \@title}\\[5ex]
            \textcolor{white}{\Large \@author}\\[2ex]
            \textcolor{white}{\@date}\\[11ex]
        \end{center}
    \vspace*{\fill}
    \thispagestyle{empty}
    \newpage
    \clearpage
    \setcounter{page}{1}
}}
\makeatother


% Includes
%! Author = keygenqt
%! Date = 4/6/24

\RequirePackage{color}

\definecolor{color_bg} {HTML}{f8faff}
\definecolor{color_primary} {HTML}{5e42ff}
\definecolor{color_secondary} {HTML}{aa82ff}


% Description
\author{Виталий Зарубин}
\title{Статья}
\date{\today}

\begin{document}

    %% Page title with background
    \maketitle

    %% First section
    \section{Section}

    Это вводный абзац в начале документа.
    Это вводный абзац в начале документа.

    %% Text custom format
    \begin{verbatim}
    \begin Это вводный абзац в начале документа.
    \end Это вводный абзац в начале документа.
    \end{verbatim}

    Это вводный абзац в начале документа.
    Это вводный абзац в начале документа.
    Это вводный абзац в начале документа.
    Это вводный абзац в начале документа.
    Это вводный абзац в начале документа.
    Это вводный абзац в начале документа.
    Это вводный абзац в начале документа.

    %% Center table
    \begin{center}
        \begin{tabular}{c r @{.} l}
            Выражение с $\pi$ &
            \multicolumn{2}{c}{Значение} \\
            \hline
            $\pi$
            & 3&1416 \\
            $\pi^{\pi}$
            & 36&46
            \\
            $(\pi^{\pi})^{\pi}$ & 80662&7 \\
        \end{tabular}
    \end{center}

    Это вводный абзац в начале документа.
    Это вводный абзац в начале документа.
    Это вводный абзац в начале документа.
    Это вводный абзац для \autoref{tab:just-1}. %% Just ref to float table
    Это вводный абзац в начале документа.
    Это вводный абзац в начале документа.
    Это вводный абзац в начале документа.
    Это вводный абзац в начале документа.

    %% Float table with ref
    \begin{table}[!ht]
        \begin{tabularx}{\textwidth}{|X|X|X|}
            \hline
            \multicolumn{3}{|c|}{My table}   \\ \hline
            0         & 2        & 4         \\ \hline
            1         & 3        & 5         \\ \hline
        \end{tabularx}
        \caption{Ref table}\label{tab:just-1}
    \end{table}

    Это вводный абзац в начале документа.
    Это вводный абзац в начале документа.
    Это вводный абзац в начале документа.
    Это вводный абзац в начале документа.

    %% Sub first section
    \subsection{Subsection}

    This is new article text.
    Это вводный абзац в начале документа.
    Это вводный абзац в начале документа.
    Это вводный абзац в начале документа.

    %% Quote example
    \begin{quote}
        Это вводный абзац в начале документа.
        Это вводный абзац в начале документа.
    \end{quote}

    %% Sub-sub first section
    \subsubsection{Subsubsection}

    Это вводный абзац
    \footnote{ %% Footnote example
        Текст сноски
    }
    в начале документа.
    %% Text selection
    Это вводный \underline{абзац} в начале документа.
    Это вводный \emph{абзац} в начале документа.
    Это вводный \textsf{абзац} в начале документа.

    %% Second selection
    \section{Конец}

    Это вводный абзац в начале документа.
    Это вводный абзац в начале документа.
    Это вводный абзац в начале документа.
    Это вводный абзац в начале документа.

    %% Lists example
    \begin{enumerate}
        \item This is new article text.
        \begin{itemize}
            \item This is new article text.
            \item[-] This is new article text.
        \end{itemize}
    \end{enumerate}

    Это вводный абзац в начале документа.
    Это вводный абзац в начале документа.
    Это вводный абзац в начале документа.
    Это вводный абзац в начале документа.
    Это вводный абзац в начале документа.

    %% Table of contents
    \newpage
    \tableofcontents

\end{document}
% End document
